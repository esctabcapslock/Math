\chapter{자기주도적 학습 과제 4 (6.3, 11.2, 13.1-3)}

\section{$3$차원 공간 안에 놓인 평면 $\gamma$ 위에 점 $C$를 중심으로 하고 반지름이 $r$인 원을 벡터로 나타내시오}
법선 벡터를 $n$이고 했을 때, 이 벡터에 수직이면서, 길이가 $r$이고 $C$를 지나면 된다.
그대로 식으로 옮기면 $\left\{ x \; | \; n \times (c - x) = 0\right\}$이 될 것이다.

\section{길이가 양수인 닫힌 구간 $I$에서 정의된 $f:I \to \mathbb{R}^3$, $\; r(t) = f(t)i+g(i)j+h(t) k$가 있다. $L=(L_1, L_2, L_3)$일 때, $\lim_{t \to p}r(t)=L$ $\iff$ $\lim_{t \to p}f(t)=L_1$ $\land$ $\lim_{t \to p}g(t)=L_2$ $\land$ $ \lim_{t \to p}h(t)=L_3 $임을 보여라}

\paragraph{$\Rightarrow$ 증명}
가정에 의해서, $\forall \epsilon >0$, $\exists \delta > 0$, $|t-p|<\delta \Rightarrow |r(t)-L|<\epsilon$이 성립함.

즉, $$\sqrt{\left(f(t)-L_1 \right)^2 + \left(f(t)-L_2 \right)^2 + \left(f(t)-L_3 \right)^2 } < \epsilon$$이 성립한다는 말이다.
그러므로, 제곱근 안에 각각의 제곱항들은 $\epsilon^2$보다 작을 것이다.

따라서, $\left|f(t)-L_1 \right| < \epsilon$,  $\left|f(t)-L_2 \right| < \epsilon$,  $\left|f(t)-L_3 \right| < \epsilon$이 성립한다.

\paragraph{$\Leftarrow$ 증명}

가정에 의해서, $\forall \frac{\epsilon}{2} >0$, $\exists \delta > 0$, $\left|f(t)-L_1 \right| < \frac{\epsilon}{2}$,  $\left|f(t)-L_2 \right| < \frac{\epsilon}{2}$,  $\left|f(t)-L_3 \right| < \frac{\epsilon}{2}$이 성립함.

즉, $$ |r(t)-L| = \sqrt{\left(f(t)-L_1 \right)^2 + \left(f(t)-L_2 \right)^2 + \left(f(t)-L_3 \right)^2 } < \sqrt{3\left(\frac{\epsilon}{2}\right)^2} < \epsilon$$이 성립한다는 말이다.

따라서 증명이 끝났다.

\section{매끄러운 곡선, 조각마다 매끄러운 곡선 뜻}
\paragraph{토머스 6.3의 정의} $f(x)$가 주어진 구간에서 연속이고 미분 가능하면 매끄러운 것임.

\paragraph{매끄러운 곡선, 토머스 13.1의 정의} 곡선 $r(t)$에 대해, 정의역의 모든 점에서 $r(t)$와 $\frac{dr}{dt}$가 연속이고, $\frac{dr}{dt}>0$
\paragraph{조각마다 매끄러운 곡선, 토머스 13.1의 정의} 매끄러운 곡선을 연결한 것.

\section{곡선의 길이 정의. 정의역의 길이가 무한대인 열린 구간인 매끄러운 곡선의 길이가 유한일 수 있을까}
\paragraph{곡선의 길이, 토마스} 매끄러운 곡선 $r(t) = x(t)i+y(t)j + z(i)k$에서, 곡선의 길이 $L$은  
$$\int_a^b \sqrt{ \left( \frac{dx}{dt} \right)^2 + \left( \frac{dy}{dt} \right)^2 + \left( \frac{dz}{dt} \right)^2 }$$
로 정의된다.

그리고, 조각마다 매끄러운 곡선의 경우, 조각들의 길이의 합으로 정의하면 될 것 같다.

\paragraph{길이를 갖는 곡선(rectifiable curve)}
해석학적으로는, 길이를 다른 방법으로 정의할 수 있다. 
구간 $I = \left[a,b\right]$를 분할한 것을 $P = \left\{ a=x_0, x_1 \cdots x_n=b \right\}$이라고 하자. $r : I \
to \mathbb{R}^n$이 때,
$||p|| \to 0$일 때, $$\sum_{k=0}^{n-1} d(x_k, x_{k+1})$$의 상한을 길이로 정의할 수 있다. 

(삼각부등식에 의해서, 노름이 0에 수렴할 경우, 주어진 식은 커지게 된다.)

\paragraph{정의역이 무한대인 유한직선}
만약 무한한 분할을 잡을 수 있다고 한다면,
$I = [0, \infty] $이고, $r(t) = \left\langle \tan^{-1}(x) \right\rangle$로 잡으면,
$$\lim_{n \to \infty} \sum_{k=0}^{infty} tan^{-1}(\frac{k+1}{n}) - tan^{-1}(\frac{k}{n}) = \lim_{m \to \infty} tan^{-1}(m) - tan^{-1}(0) = \frac{\pi}{2} $$
이니까 유한하지 않을까?

$\mathrm{Card}([0,1]) = \mathrm{Card}([0,\infty])$가 아닌가.

\section{13.3 - 13.4 학습지 13번 문제 참조}

$1 over {n \pi}$꼴로 분할을 잡은 뒤, 극대, 극소값들을 잘 연결해서 위로 유계임을 보이면 된다. 수업시간에 했으니 생략.

(참고로 답은 $p>1$이다.)

