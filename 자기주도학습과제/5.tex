\chapter{자기주도적 학습 과제 5 (13.4-5, 14.1-3)}
\section{곡선의 곡률(curvature)과 비틈림률(torsion)의 정의}

\paragraph{곡률}
곡선 $r(t)$에 대해, 단위접벡터 $T$는 $\frac{dr / dt }{ \left| dr / dt\right|}$로 정의한다.
이는, 곡선에 접하는 원(접촉원)의 반지름의 역수와 같다.

그리고 이 곡선의 곡선길이 매개변수를 $s$라고 할 때, 곡률 $\kappa$는 $\left| \frac{dT}{ds}\right|$로 정의한다.

\paragraph{비틀림률}
주 단위 접선 벡터를 $N = \frac{1}{\kappa}\frac{dT}{ds}$로 정의하고, 종 법선벡터를 $B = T \times N$로 정의한다. 이때, 비틀림률 $\tau$는 $\tau = -\frac{dB}{ds} \cdot N $로 정의한다.
이는, $T$와 $B$가 이루는 평면에서 얼마나 곡선이 벌어지는지를 나타난다. (공간적으로 얼마나 휘어 있는지를 나타낸다.)

\section{열린 집합, 닫힌 집합, 연결 집합의 정의, 집합의 연산 관련 성질 }
\paragraph{내부점 (쌤 ppt)}
$G \subset \mathbb{R}^n$과, $p \in G$에 대해서, $\exists \delta > 0 $ s.t. $\left\{ x \in \mathbb{R}^n \; | \; \left|x-p\right|<\delta \right\} \subset G$면, $p$를 $G$의 내부점이라 한다.



\paragraph{경계점 (쌤 ppt)}
$\forall \delta > 0 $ s.t. $\left\{ x \in \mathbb{R}^n \; | \; \left|x-p\right|<\delta \right\} \cap G \ne \varnothing$고, $\left\{ x \in \mathbb{R}^n \; | \; \left|x-p\right|<\delta \right\} \cap G^c \ne \varnothing$면, $p$를 $G$의 경계점이라 한다.

\paragraph{열린집합 (쌤 ppt)}
자신의 모든 점들이 내부점인 집합

\paragraph{닫힌집합 (쌤 ppt)}
자신의 경계점들을 모두 원소로 갖고 있는 집합

\paragraph{성질 (쌤 ppt)}
열린집합, 혹은 닫힌집합끼리는 유한번의 합집합, 교집합 연산에 대해 닫혀있다.
그러나, 열린집합은 무한 합집합만 열린집합이며, 닫힌집합은 무한 교집합만 닫힌집합니다.

\paragraph{연결 공간 ( 한국어 위키피디아)}
공집합이 아닌 두 열린집합으로 쪼갤 수 없는 집합이다.

또, $x,y \in X$인 집합 $X$내 연속함수 $f: [0,1] \to X$가 존재해 $f(0)=x$, $f(1)=y$를 만족하면 거리연결공간이라고 한다.
유클리드 공간의 경우, 연결 공간과 거리연결공간은 필요충분조건임이 증명되어있다.

\paragraph{연결 공간의 연산}
연결공간의 합집합은 연결공간이 되지 않을 수 있다. $\left\{ (x,y) |  x^2 + y^2 < 1 \right\}$과  $\left\{ (x,y) |  (x-3)^2 + y^2 < 1 \right\}$ 각각은 연결 공간인데, 합집합은 연결 공간이 아니다. (쪼개는 것이 가능하다.)

연결공간의 교집합도 연결공간이 되지 않을 수 있다. 두 집합의 교집합이 서로 떨어져 있을 수 있기 때문이다.

\section{$U \subset \mathbb{R}^2$은 는 공집합이 아닌 열린 연결 집합이다. $f: U \to \mathbb{R}$일 때, 점 $p \in U$에서 $f$의 편미분의 정의}
\paragraph{$x$에 관한 편미분}
$$\left.\frac{\partial f }{\partial x} \right|_{p} = \lim_{h \to 0} \frac{f(p+hi)-f(p)}{h}$$로 정의한다. 이는, $x$축에 평행하게 함수를 잘랐을 대, 이 함수의 기울기라고 할 수 있다.
$y$에 대해서도 대징적으로 성립한다.
\section{편미분가능하면 연속인가. 어떤 조건이 더 필요한가}

\paragraph{반례}
당연히 편미분 가능하면 연속이 아니다. $f(x,y)$를 다음과 같이 정의하면, 원점에서 연속이 아닌 것이 자명하다.
$$f(x) = \begin{cases}
    1, & xy=0 \\
    0, & xy \ne 0
\end{cases}
$$로 정의하면 된다.

\paragraph{추가조건}
심지어 방향미분계수가 존재해도 연속이 아니다. 음...

\section{이변수 함수의 미분계수는 몇차원인가}
실수라면, 2개의 값으로 표현해야 할 것 같다. 어느 방향으로 기울어졌는지에 관한 정보가 필요하기 때문이다. 