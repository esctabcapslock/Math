\documentclass[oneside]{oblivoir}
\usepackage{kotex}

% 뭐임
\usepackage{amsmath}
\usepackage{amssymb}
\usepackage{amsthm}
\usepackage{graphicx}
\usepackage{mathtools}
\usepackage{ruby}
%\usepackage{CJKutf8}
%https://tex.stackexchange.com/questions/95729/typesetting-furigana-above-and-below-original-text#
%--------- 멋대로 정의해본 명령어

% 여잭백 조절
\usepackage{fapapersize}
\usefapapersize{210mm, 297mm,30mm,*,35mm,30mm}


\newcommand{\dx}[1]{\operatorname{d}\! #1}
\newcommand{\term}[1]{\textbf{#1}}

\title{선대 :: 정의 복습}
\author{선대를 공부하는 사람. }
%\makeindex
\begin{document}
\maketitle
\begin{abstract}
선대 정의 정리하기. 두둥.
\end{abstract}

\newpage
%\tableofcontents

\setcounter{section}{0}
\section{집합과 함수}
\subsection{함수}
\begin{itemize}
    \item 단사함수, one-to-one, injective, $f(s)=f(s') \Rightarrow s=s'$
    \item 전사함수, onto, surjective, $\mathrm{Im}(f)=T$
    \item 전단사함수, one-to-one correspondence, bijective
\end{itemize}

\subsection{치환}


\section{군과 체}
\subsection{군}
집합 $G$와 이항연산 $\ast$에 대해, 군 $\left\langle G,\ast \right\rangle$은 다음 조건을 만족
\begin{enumerate}
    \item $G$는 $\ast$에 닫혀있음
    \item $\ast$는 \ruby{결합법칙}{associative} 만족
    \item 항등원 존재
    \item 모든 원소에 역원 존재 ($s \ast t = t \ast s = e $)
\end{enumerate}

\subsection{환}
집합 $R$과 연산 $+$, $\cdot$에 대해서 다음을 만족
\begin{enumerate}
    \item $\left\langle R, + \right\rangle$는 덧셈군(=가환)
    \item $\left\langle R, \cdot \right\rangle$는 반군 (덧셈, 결합법칙 有)
    \item 분배법칙 성립. $a(b+c)=ab+ac$
\end{enumerate}

\subsection{체}
다음을 만족하는 환 $K$를 체라고 함.
\begin{enumerate}
    \item 가환환 (commutative ring), ($=$교환법칙 성립)
    \item ring with unity ($\cdot$의 항등원 존재)
\end{enumerate}


\section{벡터공간과 선형변환}
\subsection{벡터공간}
체 $K$, 덧셈군 $V$에 대해,
 $K \times V \to V$인 스칼라 연산이 존재.
 그리고 다음을 만족
\begin{enumerate}
    \item $\left\langle  V,+ \right\rangle$는 가환군
    \item $(\lambda \mu)v = \lambda(\mu v) $
    \item $(\lambda + \mu) v =\lambda v + \mu v$
    \item $\lambda (v + w) = \lambda v + \lambda w$
    \item $1 v = v$
\end{enumerate}

\subsection{부분공간}
$W$가 $V$의 부분공간일 필요충분조건은, $W$가 스칼라연산과 덧셈연산에 대해 닫혀있다는 것이다.

\subsection{선형변환}
같은 체 $K$위에 정의된 두 벡터공간 $V$, $V'$에 대해,
함수 $T : V \to V'$이 다음을 만족하면 선형 변환임.
\begin{enumerate}
    \item $T(v+w) = T(v)+T(w)$
    \item $T(\lambda v) = \lambda T(v)$
\end{enumerate}
%\begin{itemize}
%\end{itemize}

\section{차원}
\subsection{랭크-널리티 정리}
선형 변환 $T : V \to W$에 대하여\dots
\begin{itemize}
    \item $\mathrm{dim} (\mathrm{Im} (T)) = \mathrm{rk} (T)$
    \item $\mathrm{dim} (\mathrm{Ker} (T)) = \mathrm{null} (T)$
    \item $\mathrm{rank} (T)+\mathrm{nullity} (T)= \mathrm{dim} (V)$
\end{itemize}


\section{행렬}
\subsection{행렬의 성분의 표현}
$$\textbf{A}_i^j = a_{ij} (\text{옳은 표현인지 모름})$$ 
$$e_i^T \textbf{A} = A_i$$
$$\textbf{A} e_j = A^j$$


\subsection{계수행렬과 첨가행렬}
$ Ax=B$에서, 
\begin{itemize}
    \item $A$는 \ruby{\term{계수행렬}}{coefficient matrix}
    \item $\left(A|B\right)$는 \ruby{\term{첨가행렬}}{augmented matrix}
\end{itemize}



\section{선형변환과 행렬표현}
\subsection{선형대수학의 기본정리(가제)}
실은 둘이 같음. 일단 선형 변환의 공간을 만듦.

$\mathrm{Hom}(K^n, K^m)$에서 논의한다. $T$에 대해 $M(T)$를 정의하고, 서로 동형이며, 선형변환의 합성은 행렬의 곱임.

$\mathrm{Hom}(V, V')$로 확장함. 좌표사상(coordinate map)을 이용.
\subsection{쌍대공간}
$rk(T)=rk(T^*)$, $M(T) = M(T^*)^T$, 우역원 존재시 좌역원 존재. 이 둘은 동일!!!

\subsection{기저의 변환}


\section{내적공간}
\subsection{실내적공간}
사상 $V \times V \to \mathbb{R}, \left( v, w, \right) \mapsto \left\langle v | w \right\rangle$이 정의된
벡터공간 $V$
\begin{enumerate}
    \item $\left\langle v | v \right\rangle \ge 0 $이며, 등호는 $v=0$때만
    \item $\left\langle v | w \right\rangle = \left\langle w | v \right\rangle$
    \item \begin{itemize}
        \item $\left\langle u+v | w \right\rangle = \left\langle u | w \right\rangle + \left\langle v | w \right\rangle$
        \item $\left\langle av | w \right\rangle = a\left\langle v | w \right\rangle$
    \end{itemize}
\end{enumerate}

\subsection{통상내적}
$$\left\langle x | y \right\rangle = \sum_{i=1}^{n} x_i y_i $$

\subsection{복소내적공간}
사상 $V \times V \to \mathbb{C}, \left( v, w, \right) \mapsto \left\langle v | w \right\rangle$이 정의된
벡터공간 $V$
\begin{enumerate}
    \item $\left\langle v | v \right\rangle \ge 0 $이며, 등호는 $v=0$때만
    \item $\left\langle v | w \right\rangle = \overline{ \left\langle w | v \right\rangle }$
    \item \begin{itemize}
        \item $\left\langle u+v | w \right\rangle = \left\langle u | w \right\rangle + \left\langle v | w \right\rangle$
        \item $\left\langle av | w \right\rangle = a\left\langle v | w \right\rangle$
    \end{itemize}
   
\end{enumerate}
마지막 2개는 \ruby{\term{반선형성}}{antilinearity}이라 함


\section{행렬식}
다음을 만족하는 사상 $\mathrm{det} : M_n (K) \mapsto K$ 을 행렬식이라 함
\begin{enumerate}
    \item $A_j = \lambda_1 C_1 + \lambda_2 C_2$, $\mathrm{det}(A) $ $=$ $ \lambda_1 \mathrm{det} (A^1 \cdots C_1 \cdot A^n) + \lambda_2 \mathrm{det} (A^1 \cdots C_2 \cdots A^n)$
    \item $A = (A^1 A^2 \cdots A^n)$에서, $A^j = A^{j+1}$면, $\mathrm{det}(A)=0$
    \item $\mathrm{det} (I_n) = 1$
\end{enumerate}
각각  \ruby{다중선형성}{Multilinearity}, \ruby{부호 교대성질}{Alternation of Sign}, \ruby{정규화 성질}{Normalization}이라 불림

\subsection{행렬식의 표현}
$$\mathrm{det} (A) = \sum_{j=1}^{n} (-1)^{ i+j } a_{ij} \mathrm{det}(\partial_{ij} A)$$
$$\mathrm{det} (A) = \sum_{\pi \in S_n } \sigma(\pi) a_{\pi(1)1} \cdots  a_{\pi(n)n}$$
후자를 이용해, $\mathrm{det} (A) = \mathrm{det} (A^{T})$임을 보일 수 있다.

\subsection{여인자}
\begin{itemize}
    \item $C_{ij} = (-1)^{i+j} \mathrm{det} (\partial_{ij}A)$ 인 $C$를 \ruby{\term{여인자}}{cofactors}라고 부름.
    \item $C_i^j$성분이 $C_{ij}$인 $n \times n$행렬 $C$를 \term{여인자 행렬}이라 부르며,
    \item 이것의 전치행렬을 \ruby{\term{수반행렬}}{adjoint matrix}이라 부름.
    \item 특히, $A^{-1} = \frac{1}{\mathrm{det}(A)} \mathrm{adj}(A)$가 성립함.
    
\end{itemize}

\section{교윳값, 고유벡터}
\subsection{정의}
$T : V \to T$인 선형사상에 대해 $0$아닌 $v \in V$에 대해 $T(v) = \lambda v$를 만족시킬 때,
$\lambda$를 \ruby{\term{고윳값}}{eigenvalue}, $0$아닌 벡터 $v$를 \ruby{\term{고유벡터}}{eigenvector}라 함.
\subsection{특성다항식}
$$p(t) = \mathrm{det}(t I_n - A)$$


\section{자기준동형사상의  표현}

\section{부록: $\mathrm{det}(A) \ne 0$과 동치들}
$A \in M_n (K)$일때, 다음은 동치임.
\begin{enumerate}
    \item $\forall y \in K$, $A\mathbf{x}=\mathbf{y}$의 해는 최소 하나 이상
    \item $A$의 행공간은 $K^n$
    \item $A$의 열공간은 $K^n$
    \item $A\mathbf{x}=\mathbf{0}$은, 자명해 $\mathbf{x}=\mathbf{0}$만 갖음
    \item $A$의 행들은 일차독립
    \item $A$의 열들은 일차독립
    \item $\forall y \in K$, $A\mathbf{x}=\mathbf{y}$는 무조건 하나의 해 갖음
    \item $A$의 행들은 $K^n$의 기저
    \item $A$의 열들은 $K^n$의 기저
    \item $A \in \mathrm{GL}_n(K)$
    \item $A^T \in \mathrm{GL}_n(K)$
    \item $\mathrm{det}(A) \ne 0$
    \item $\mathrm{det}(A^T) \ne 0$
\end{enumerate}

\end{document}